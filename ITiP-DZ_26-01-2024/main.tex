\documentclass{article}
\usepackage[T1]{fontenc}
\usepackage[croatian]{babel}
\usepackage[utf8]{inputenc}

\usepackage{graphicx} % Paket za umetanje slika
\usepackage{tabularx} % Paket za napredne tablice
\usepackage{amsmath} % Paket za matematiku
\usepackage{hyperref} % Paket za hyperlinkove
\usepackage{float} % Za bolje pozicioniranje slika

\title{Konfiguracija gaming računala}
\author{Zdenko Gega}
\date{Siječanj 2024}

\setcounter{secnumdepth}{1}

\begin{document}

\maketitle
\clearpage
\tableofcontents
\clearpage

\section{Uvod}
Uvodni tekst
\clearpage

\section{Komponente}

    \subsection{Procesor (CPU)}
    Izabran je procesor \textit{Intel Core i5-13400} koji je dio \textit{Intel Raptor Lake-S} serije Intel procesora. Ovo je procesor 13.-e generacije. Podnožje \emph{(eng. socket)} procesora je \textit{Intel LGA1700}. Procesor ima 10 jezgri i 16 niti \emph{(eng. threads)}. Ima 20MB L3 predmemorije \emph{(eng. Cache)} Intel Smart Cache, te 9.5MB L2 Cache. Osnovni radni takt procesora je 2.50GHz, a maksimalni radni takt procesora je 4.60GHz. Procesor zahtjeva 65W u normalnom radu, te do 154W u maksimalnom modu rada. Tehnologija prozivodnje 10nm. Procesor ima ugrađen i grafički procesor \textit{Intel UHD Graphics 730} koji ima raspon rada od 300MHz do 1550MHz. Na testiranju procesora \emph{(eng. Benchmark)} u Multi Thread načinu rada sa četiri prolaska, program pokazuje 59510 MIPS \emph{(eng. mega instructions per second)} što pokazuje brzinu izvedbe naredbi u sekundi. Izabrao sam ovaj procesor, jer se nalazi među vrhom najboljih procesora za \textit{Gaming}. U omjeru performansi i cijene.

    \begin{figure}[H]
        \centering
        \includegraphics[width=0.5\textwidth]{Slike/Procesor.jpg}
        \caption{Procesor Intel Core i5-13400}
        \label{fig:Procesor}
    \end{figure}

    \begin{table}[H]
        \centering
        \begin{tabular}{|c|c|}
            \hline
            Naziv & Cijena \\
            \hline
            Intel Core i5-13400 & 299 € \\
            \hline
        \end{tabular}
        \caption{Cijena procesora}
        \label{tab:Procesor}
    \end{table}
    
    \clearpage
    \subsubsection{Hladnjak procesora}
    Izabran hladnjak za procesor je \textit{Zalman CNPS10X Optima II}. Hladnjak na sebi ima ventilator veličine 120mm sa RGB osvjetljenjem za estetiku. Sam hladnjak se sastoji od čistog bakra i aluminija. Za ovu komponentu nema nekog specifičnog razloga zašto sam je odabrao, osim što je bolja nego bilo koji hladnjak koji dolazi u originalnom pakiranju, te ujedno zbog estetskih razloga, jer posjeduje RGB ventilator.

    \begin{figure}[H]
        \centering
        \includegraphics[width=0.5\textwidth]{Slike/CPU_hladnjak.jpg}
        \caption{CPU hladnjak Zalman CNPS10X}
        \label{fig:Hladnjak}
    \end{figure}

    \begin{table}[H]
        \centering
        \begin{tabular}{|c|c|}
            \hline
            Naziv & Cijena \\
            \hline
            Zalman CNPS10X Optima II & 44 € \\
            \hline
        \end{tabular}
        \caption{Cijena hladnjaka za CPU}
        \label{tab:Hladnjak}
    \end{table}

    \clearpage
    \subsection{Matična ploča}
    Izabrana matična je \textit{ASRock B660M Pro} sa odgovarajućim \textit{Intel LGA1700} podnožjem. Matična ploča je \textit{Micro ATX} i podržava dvokanalnu DDR4 memorijsku tehnologiju, te ima četiri puta DDR4 DIMM utora do maksimalno 128GB. Ima dva PCIe x16 utora, jedan PCIe Gen3 utor, te jedan M.2 Utor. Što se tiče pohrane, na njoj se još također nalazi četiri puta SATA3 (6.0Gb/s) konektora, te jedan Hyper M.2 utor (64Gb/s) i jedan Ultra M.2 utor (32Gb/s). Ima 7.1 kanalni HD audio ulaz/izlaz. Ima integriranu mrežnu karticu koja podržava Gigabit LAN 10,100,1000 Mb/s. Posjeduje četiri USB 3.2 priključka i dva USB 2.0 priključka. Ova matična je izabrana prvenstveno zbog podnožja za procesor koji je potreban, te zbog same marke \textit{ASRock} koja je dosta poznata po svojoj kvaliteti duži niz godina. Ujedno osobno imam dobra iskustva sa navedenom markom.

     \begin{figure}[H]
        \centering
        \includegraphics[width=0.5\textwidth]{Slike/Maticna_ploca.jpg}
        \caption{Matična ploča ASRock B660M Pro}
        \label{fig:Maticna}
    \end{figure}

    \begin{table}[H]
        \centering
        \begin{tabular}{|c|c|}
            \hline
            Naziv & Cijena \\
            \hline
            ASRock B660M Pro & 132,12 € \\
            \hline
        \end{tabular}
        \caption{Cijena matične ploče}
        \label{tab:Maticna}
    \end{table}

    \clearpage
    \subsection{Memorija (RAM)}
    Odabrana memorija \emph{(eng. Random Access Memory, RAM)} je \textit{Kingston Fury Beast XMP}. Vrsta memorije je DDR4 koja paše na izabranu matičnu ploču. Memorija je brzine od 3200MHz, te je uzeto dva puta po 16GB kapaciteta memorije. Latencija je CL16. Svaki modul je testiran za rad na DDR4-3200MHz uz nisku latenciju. Ova memorija također ima tri profila rada. Standardni profil je na 2400MHz sa 1.2V, zatim dva XMP profila, od kojih jedan na 3000MHz sa 1.35V, a drugi sa 3200MHz na 1.35V. Profili se mogu izabrati u postavkama BIOS-a. Memorija također na sebi ima metalno kućište koje doprinosi pasivnom hlađenju memorije tijekom rada. Ova memorija je izabrana zbog svojeg omjera dobre performanse i cijene, te same marke Kingston, koja je također poznata marka i osobno imam dobro dugogodišnje iskustvo ove marke.

    \begin{figure}[H]
        \centering
        \includegraphics[width=0.5\textwidth]{Slike/RAM_kingston.jpg}
        \caption{Memorija (RAM) Kingston Fury}
        \label{fig:Memorija}
    \end{figure}

    \begin{table}[H]
        \centering
        \begin{tabular}{|c|c|}
            \hline
            Naziv & Cijena \\
            \hline
            Kingston Fury Beast XMP & 91,27 € \\
            \hline
        \end{tabular}
        \caption{Cijena RAM-a}
        \label{tab:Memorija}
    \end{table}

    \clearpage
    \subsection{Grafička kartica (GPU)}
    Odabrana grafička kartica je \textit{Gigabyte GeForce RTX3060 Windforce OC}. Serija grafičke kartice je \textit{GeForce RXT3060} sa 12GB GDDR6 memorije. Radni takt jezgre je 1792MHz, a radni takt memorije je 15000MHz. Memorijska sabirnica je 192bit-na. Širina pojasa \emph{(eng. Bandwith)} memorije je 360GB/s. Maksimalna rezolucija grafičke kartice je 1680x4320, te podržava \textit{DirectX 12 Ultimate} i \textit{OpenGL 4.6}. Grafička kartica ima dva DisplayPort-a 1.4a, te dva HDMI 2.1 konektora. Preporučeno napajanje za ovu grafičku karticu je 550W, dok mi imamo izabrano napajanje od 650W što je više nego dovoljno, dok uzmemo u obzir procesor i ostale komponente.

    \begin{figure}[H]
        \centering
        \includegraphics[width=0.5\textwidth]{Slike/Graficka_kartica.jpg}
        \caption{Grafička kartica (GPU) GeForce RTX3060}
        \label{fig:Graficka}
    \end{figure}

    \begin{table}[H]
        \centering
        \begin{tabular}{|c|c|}
            \hline
            Naziv & Cijena \\
            \hline
            Gigabyte GeForce RTX3060 Windforce OC & 375 € \\
            \hline
        \end{tabular}
        \caption{Cijena grafičke kartice}
        \label{tab:Graficka}
    \end{table}

    \clearpage
    \subsection{Pohrana podataka}
    Za pohranu podataka izabran je SSD disk \emph{(eng. Solid State Disk)}, proizvođača \textit{Western Digital}. Model je SN350, sa prostorom za pohranu podataka od 1TB. Sučelje SSD-a je M.2 NVMe PCIe Gen3. Brzina čitanja podataka je 3200MB/s, a brzina zapisivanja podataka 2500MB/s. SSD u usporedbi sa običnim SATA tvrdim diskom, ima puno veće brzine čitanja i zapisivanja podataka. Pored proizvođača \textit{Western Digital} mogu preporučiti proizvođače \textit{Seagate i Kingston}, naime izabran je \textit{Western Digital} zbog dobrog osobnog iskustva kroz dugogodišnje korištenje SATA tvrdih diskova ovog proizvođača.

    \begin{figure}[H]
        \centering
        \includegraphics[width=0.5\textwidth]{Slike/SSD_NVMe.jpg}
        \caption{SSD WD Green SN350}
        \label{fig:Pohrana}
    \end{figure}

    \begin{table}[H]
        \centering
        \begin{tabular}{|c|c|}
            \hline
            Naziv & Cijena \\
            \hline
            Western Digital SN350 & 85,02 € \\
            \hline
        \end{tabular}
        \caption{Cijena SSD-a}
        \label{tab:Pohrana}
    \end{table}

    \clearpage
    \subsection{Napajanje (PSU)}
    Napajanje \emph{(eng. Power Supply Unit, PSU)} je izabrano \textit{Corsair CX750}, jačine 750W. Napajanje nije modularno, što znači da nema mogućnost uklanjanja viška kablova za \textit{cable management} ali sadrži sve bitne konektore koji su potrebni za spajanje ostalih komponenti. Učinkovitost napajanja je 80\texttt{+} Bronze. Ima radnu učinkovitost do 88\%, te stvara manje topline i smanjuje troškove električne energije. Opremljeno je 120mm toplinski kontroliranim ventilatorom koji se okreće velikom brzinom samo kada je napajanje pod velikim opterećenjem, automatski usporavajući za tiši rad pri manjim opterećenjima. Ovo napajanje je izabrano iz razloga što je proizvođač \textit{Corsair} dosta poznat po svojim kvalitetnim napajanjima, te iz razloga što je dosta recenzija na internetu pokazalo da je napajanje dobro u smislu kvalitete i cijene.

    \begin{figure}[H]
        \centering
        \includegraphics[width=0.5\textwidth]{Slike/PSU_corsair.jpg}
        \caption{Napajanje Corsair CX650}
        \label{fig:Napajanje}
    \end{figure}

    \begin{table}[H]
        \centering
        \begin{tabular}{|c|c|}
            \hline
            Naziv & Cijena \\
            \hline
            Corsair CX750 & 91,29 € \\
            \hline
        \end{tabular}
        \caption{Cijena Napajanja}
        \label{tab:Napajanje}
    \end{table}

    \clearpage
    \subsection{Kućište}
    Izbor kućišta je bitan ali ne i ključan u sastavljanju svih komponenti. Najbitnije je paziti na veličinu kućišta, zbog matične ploče. U ovom slučaju iazbrao sam kućište \textit{Bit Force Paladin ARGB-4} koje je srednje gaming kućište \emph{(eng. Mid Tower)}, te podržava našu izabranu mATX matičnu ploču, ujedno ostavlja i prostora za nadogranju na ATX matičnu ploču. Dolazi sa kaljenim staklom i ugrađenim 4x120mm ARGB ventilatorima, također ima podršku za vodeno hlađenje. Kućište ima jedan USB 3.0 utor, te dva USB 2.0 utora i HD audio ulaze. Sa prednje strane se nalaze tri ventilatora te ispred njih je aluminijska metalna mreža. Izbor kućišta je više stvar preferencije pojedinca, jer svatko ima drugačiji izbor.

    \begin{figure}[H]
        \centering
        \includegraphics[width=0.5\textwidth]{Slike/Kuciste.jpg}
        \caption{Kućište Bit Force Paladin ARGB-4}
        \label{fig:Kuciste}
    \end{figure}

    \begin{table}[H]
        \centering
        \begin{tabular}{|c|c|}
            \hline
            Naziv & Cijena \\
            \hline
            Bit Force Paladin ARGB-4 & 63,28 € \\
            \hline
        \end{tabular}
        \caption{Cijena kućišta}
        \label{tab:Kuciste}
    \end{table}

    \clearpage
    \subsection{Monitor}
    Izabran je monitor \textit{Asus TUF Gaming VG279Q1A}. Ovaj monitor je veličine 27" dijagonale. Ima IPS panel, sa osvježenjem zaslona od 165Hz. IPS panel ima odlične boje i sliku, te odličan kontrast i izvrstan kut gledanja. Ima odaziv od 1ms, što je vrlo dobro za brze igre. Rezolucija monitora je 2560x1440 sa omjerom od 16:9. Također ima HDR10 podršku, što čini igranje igrica na ovom monitoru nezaboravljivim iskustvom. Ima tehnologiju ultra niskog plavog osvjetljenja, što uvelike smanjuje loš učinak plavog svjetla. Uz ovaj monitor vrlo dobro ide DisplayPort koji je uobičajen na grafičkim karticama \textit{RTX3060}, kao što je gore izabrana. Ovaj monitor je izabran zbog svojih dobrih karakteristika i ujedno pristupačne cijene.

    \begin{figure}[H]
        \centering
        \includegraphics[width=0.5\textwidth]{Slike/Monitor.jpg}
        \caption{Monitor Asus TUF Gaming}
        \label{fig:Monitor}
    \end{figure}

    \begin{table}[H]
        \centering
        \begin{tabular}{|c|c|}
            \hline
            Naziv & Cijena \\
            \hline
            Asus TUF Gaming VG279Q1A & 175 € \\
            \hline
        \end{tabular}
        \caption{Cijena monitora}
        \label{tab:Monitor}
    \end{table}

    \clearpage
    \subsection{Miš}
    Miš je \textit{Logitech G502 Hero}. Ovaj miš je među boljim ergonomičnim gaming miševima, ujedno \textit{Logitech} marka je poznata po svojoj kvaliteti. Miš je žičani, ima raspon od 100 do 25600 točaka po inču \emph{(eng. Dots Per Inch, DPI)}. Također ima 11 tipki koje se mogu programirati da rade razne funkcije. Miš se može izabrati i neki bežični ali opet je to stvar preferencije pojedinca. Također ima RGB osvjetljenje koje se može sinkronizirati sa raznim igra, što je samo više za estetski izgled.

    \begin{figure}[H]
        \centering
        \includegraphics[width=0.5\textwidth]{Slike/Mis.jpg}
        \caption{Miš Logitech G502 Hero}
        \label{fig:Mis}
    \end{figure}

    \begin{table}[H]
        \centering
        \begin{tabular}{|c|c|}
            \hline
            Naziv & Cijena \\
            \hline
            Logitech G502 Hero & 49,99 € \\
            \hline
        \end{tabular}
        \caption{Cijena miša}
        \label{tab:Mis}
    \end{table}

    \clearpage
    \subsection{Tipkovnica}
    Tipkovnica je \textit{AOC AGK700}, serije \textit{Agon}. Ovo je mehanička RGB tipkovnica, sa adresabilnim tipkama, što znači da se putem određenog softvera \emph{(eng. Software)} može svaka tipka zasebno podesiti, što da radi. Tipkovnica ima \textit{Cherry MX Red switches}. Ove crvene tipke su jedna od vodećih karakteristika kada je u pitanju vrhunska kvaliteta na tržištu igara. Nevjerojatno brz odziv, potrebno je 45 grama pritiska za aktiviranje bilo koje tipke. Tipkovnica ima funkcije \textit{N-Key rollover \& 100\% Anti-Ghosting}. Dok igrica zahtjeva više akcija u minuti \emph{(eng. Actions Per Minute, APM)}, tu dolazi do izražaja \textit{N-Key Rollover}, koji osigurava da se sav unos ispravno registrira. Tipkovnica ima 5 tipki određenih za makro naredbe, dodatne multimedijske tipke, te USB 2.0 priključak na samoj tipkovnici. Izbor tipkovnice je zbog dobrih karakteristika i recenzija, te zbog osobnog iskustva sa tipkovnicom.

    \begin{figure}[H]
        \centering
        \includegraphics[width=0.5\textwidth]{Slike/Tipkovnica.jpg}
        \caption{Tipkovnica AOC AGK700}
        \label{fig:Tipkovnica}
    \end{figure}

    \begin{table}[H]
        \centering
        \begin{tabular}{|c|c|}
            \hline
            Naziv & Cijena \\
            \hline
            AOC AGK700 & 109,99 € \\
            \hline
        \end{tabular}
        \caption{Cijena tipkovnice}
        \label{tab:Tipkovnica}
    \end{table}

    \clearpage
    \subsection{Slušalice}
    Izbor slušalica je \textit{Razer BlackShark V2 X}. Slušalice nisu bežične, što meni osobno ne predstavlja problem ali to može biti stvar osobnog izbora. Slušalice podržavaju 7.1 virtualni okolni zvuk \emph{(eng. Virtual Surround Sound)}, što omogućuje točnu poziciju zvuka koja omogućuje da intuitivno odredimo odakle svaki zvuk dolazi. Imaju jastučiće od prozračne memorijske pjene, što dobrinosi udobnosti kod dužeg korištenja slušalica. Imaju napredno pasivno poništavanje buke \emph{(eng. Noise Cancelling)}, te imaju jako dobru izolaciju niskih i visokih tonova. Imaju 3.5mm priključak, što znači da su kompatibiline na više vrsta uređaja.

    \begin{figure}[H]
        \centering
        \includegraphics[width=0.5\textwidth]{Slike/Slusalice.jpg}
        \caption{Slušalice Razer BlackShark V2 X}
        \label{fig:Slusalice}
    \end{figure}

    \begin{table}[H]
        \centering
        \begin{tabular}{|c|c|}
            \hline
            Naziv & Cijena \\
            \hline
            Razer BlackShark V2 X & 73,99 € \\
            \hline
        \end{tabular}
        \caption{Cijena slušalica}
        \label{tab:Slusalice}
    \end{table}

\clearpage
\section{Troškovi}

    \begin{table}[H]
        \centering
        \begin{tabular}{|c|c|}
            \hline
            Naziv & Cijena \\
            \hline
            Razer BlackShark V2 X & 73,99 € \\
            \hline
        \end{tabular}
        \caption{Cijena slušalica}
        \label{tab:Slusalice}
    \end{table}

\clearpage
\section{Zaključak}
Ukratko napisat zaključak
\clearpage

\listoffigures

\end{document}
