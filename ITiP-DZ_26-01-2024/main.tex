\documentclass{article}
\usepackage[T1]{fontenc}
\usepackage[croatian]{babel}
\usepackage[utf8]{inputenc}

\usepackage{graphicx} % Paket za umetanje slika
\usepackage{tabularx} % Paket za napredne tablice
\usepackage{amsmath} % Paket za matematiku
\usepackage{hyperref} % Paket za hyperlinkove
\usepackage{float} % Za bolje pozicioniranje slika

\title{Konfiguracija gaming računala}
\author{Zdenko Gega}
\date{Siječanj 2024}

\setcounter{secnumdepth}{1}

\begin{document}

\maketitle
\clearpage
\tableofcontents
\clearpage

\section{Uvod}
Uvodni tekst
\clearpage

\section{Komponente}
    Procesor: Intel Core i5-13400

    \subsection{Procesor (CPU)}
    Izabran je procesor \textit{Intel Core i5-13400}. Ovo je procesor 13.-e generacije. Podnožje \emph{(eng. socket)} procesora je \textit{Intel LGA1700}. Procesor ima 10 jezgri i 16 thread-ova, te 20MB predmemorije \emph{(eng. Cache)} Intel Smart Cache, te 9.5MB L2 Cache. Osnovni radni takt procesora je 2.50GHz, a maksimalni radni takt procesora je 4.60GHz. Procesor zahtjeva 65W u normalnom radu, te do 154W u maksimalnom modu rada. Izabrao sam ovaj procesor, jer se nalazi među vrhom najboljih procesora za \textit{Gaming}. U omjeru performansi i cijene.

    \begin{figure}[H]
        \centering
        %\includegraphics[width=0.5\textwidth]{Slike/Procesor.jpg}
        \caption{Caption}
        \label{fig:Procesor}
    \end{figure}
    \subsubsection{Hladnjak procesora}
    Izabran hladnjak za procesor je \textit{Zalman CNPS10X Optima II}. Hladnjak na sebi ima ventilator veličine 120mm sa RGB osvjetljenjem za estetiku. Sam hladnjak se sastoji od čistog bakra i aluminija. Za ovu komponentu nema nekog specifičnog razloga zašto sam je odabrao, osim što je bolja nego bilo koji hladnjak koji dolazi u originalnom pakiranju, te ujedno zbog estetskih razloga, jer posjeduje RGB ventilator.

    \clearpage
    \subsection{Matična ploča}
    Izabrana matična je \textit{ASRock B660M Pro} sa odgovarajućim \textit{Intel LGA1700} podnožjem. Matična ploča je \textit{Micro ATX} i podržava dvokanalnu DDR4 memorijsku tehnologiju, te ima četiri puta DDR4 DIMM utora do maksimalno 128GB. Ima dva PCIe x16 utora, jedan PCIe Gen3 utor, te jedan M.2 Utor. Što se tiče pohrane, na njoj se još također nalazi četiri puta SATA3 (6.0Gb/s) konektora, te jedan Hyper M.2 utor (64Gb/s) i jedan Ultra M.2 utor (32Gb/s). Ima 7.1 kanalni HD audio ulaz/izlaz. Ima integriranu mrežnu karticu koja podržava Gigabit LAN 10,100,1000 Mb/s. Posjeduje četiri USB 3.2 priključka i dva USB 2.0 priključka. Ova matična je izabrana prvenstveno zbog podnožja za procesor koji je potreban, te zbog same marke \textit{Gigabyte} koja je dosta poznata po svojoj kvaliteti duži niz godina. Ujedno osobno imam dobra iskustva sa navedenom markom.

    \clearpage
    \subsection{Memorija (RAM)}
    Odabrana memorija \emph{(eng. Random Access Memory, RAM)} je \textit{Kingston Fury Beast XMP}. Vrsta memorije je DDR4 koja paše na izabranu matičnu ploču. Memorija je brzine od 3200MHz, te je uzeto dva puta po 16GB kapaciteta memorije. Latencija je CL16.Svaki modul je testiran za rad na DDR4-3200MHz uz nisku latenciju. Ova memorija također ima tri profila rada. Standardni profil je na 2400MHz sa 1.2V, zatim dva XMP profila, od kojih jedan na 3000MHz sa 1.35V, a drugi sa 3200MHz na 1.35V. Profili se mogu izabrati u postavkama BIOS-a. Ova memorija je izabrana zbog svojeg omjera dobre performanse i cijene, te same marke Kingston, koja je također poznata marka i osobno imam dobro dugogodišnje iskustvo ove marke.

    \subsection{Grafička kartica (GPU)}
    Opis grafičke kartice

    \subsection{Pohrana}
    Opis prostora za pohranu

    \subsection{Napajanje (PSU)}
    Opis napanja

    \subsection{Kućište}
    Opis kućišta

\section{Cijene}
    \begin{table}[H]
        \centering
        \begin{tabular}{|l|r|}
            \hline
            Komponenta & Cijena (€)\\hline
            Procesor & X\\hline
        \end{tabular}
        \caption{Caption}
        \label{tab:my_label}
    \end{table}
\clearpage

\section{Zaključak}
Ukratko napisat zaključak
\clearpage

\listoffigures

\end{document}
